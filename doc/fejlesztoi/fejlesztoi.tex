\documentclass[fleqn,10pt,a4paper,titlepage]{article}
\includeonly{common,komplexfv}

%
% - ---- -- PACKAGES--------------------------
%

\usepackage{amssymb}
\usepackage{amsmath}
\usepackage[utf8]{inputenc}
\usepackage[magyar]{babel}
%\usepackage{amsthm}
\usepackage{t1enc}
\usepackage{theorem}
\usepackage{fancyhdr}
\usepackage{lastpage}
\usepackage{paralist}

%
% ---------- CODES --------------------------
%
\makeatletter
\gdef\th@magyar{\normalfont\slshape
  \def\@begintheorem##1##2{%
  \item[\hskip\labelsep \theorem@headerfont ##2.\ ##1.]}%
  \def\@opargbegintheorem##1##2##3{%
  \item[\hskip\labelsep \theorem@headerfont ##2. ##1.\ (##3)]}}
\makeatother


%
% ------------  N E W  C O M M A N D S --------
%
%\newcommand{\ob}{\begin{flushright} \leavevmode\hbox to.77778em{\hfil\vrule
%    \vbox to.675em{\hrule width.6em\vfil\hrule}\vrule}\hfil\end{flushright}}
\newcommand{\ob}{\hfill$\square$}
\newcommand{\ff}{f\in\mathbb{R}\rightarrow\mathbb{R}}
\newcommand{\fab}{f\colon (a,b)\rightarrow\mathbb{R}}
\newcommand{\fabk}{f\colon \left[a,b\right]\rightarrow\mathbb{R}}
\newcommand{\fir}{f\colon I\rightarrow\mathbb{R}}
\newcommand{\fdab}{f\in D(a,b)}
\newcommand{\fcab}{f\in C[a,b]}
\newcommand{\exist}{\exists}
\newcommand{\ek}{\Longleftrightarrow}
\newcommand{\la}{\lambda}
\newcommand{\ro}{\varrho}
\newcommand{\K}{\ensuremath{\mathbb{K}}}
\newcommand{\R}{\ensuremath{\mathbb{R}}}
\newcommand{\Q}{\ensuremath{\mathbb{Q}}}
\newcommand{\N}{\ensuremath{\mathbb{N}}}
\newcommand{\C}{\ensuremath{\mathbb{C}}}
\newcommand{\n}{\ensuremath{\to}} %azonos a rightarrow-val
%duplanyil, szuksegesseg
\newcommand{\nn}{\ensuremath{\Rightarrow}}
%\newcommand{\Omage}{\Omega}
%elegségesség, nem def :)
%\newcommand{\nb}{\Leftarrow}
\newcommand{\di}{\displaystyle}
\newcommand{\sarrow}{\downarrow}
\newcommand{\narrow}{\uparrow}
\newcommand{\lt}{<}
\newcommand{\gt}{>}
\newcommand{\Int}{\intop\limits}
\newcommand{\ures}{\varnothing}
\newcommand{\ekv}{\iff}
\newcommand{\ekviv}{\ekv}
\renewcommand{\epsilon}{\varepsilon}
\newcommand{\eps}{\varepsilon}
%
% ------------  NEW PART DEFS -----------------
%
\newcounter{Szaml}


\theoremstyle{magyar}
\theoremheaderfont{\itshape\bfseries}
\newtheorem{de}{definíció}[section]
\newtheorem{te}{tétel}[section]
\newtheorem{bi}{bizonyítás}[section]
\newtheorem{ko}{következmény}[section]
\newtheorem{me}{megjegyzés}[section]
\newtheorem{al}{állítás}[section]


\newenvironment{korlista}{\begin{enumerate}[\quad1$^\circ$]}{\end{enumerate}}

\newenvironment{biz}{\begin{trivlist}\item\relax\mbox{\textbf{Bizonyítás.\enskip}}\ignorespaces}{\ob\end{trivlist}}
\newenvironment{Biz}{\begin{trivlist}\item\relax\mbox{\textbf{Bizonyítás.\enskip}}\ignorespaces\begin{korlista}}{\ob\end{korlista}\end{trivlist}}
\newenvironment{kov}{\begin{trivlist}\item\relax\mbox{\textbf{Következmény.\enskip}}\ignorespaces}{\end{trivlist}}
\newenvironment{megj}{\begin{trivlist}\item\relax\mbox{\textbf{Megjegyzés.\enskip}}\ignorespaces}{\end{trivlist}}
\newenvironment{Megj}{\begin{megj}\begin{korlista}}{\end{korlista}\end{megj}}
\newenvironment{pl}{\begin{trivlist}\item\relax\mbox{\textbf{Példa.\enskip}}\ignorespaces}{\end{trivlist}}
\newenvironment{Pl}{\begin{pl}\begin{korlista}}{\end{korlista}\end{pl}}
\DeclareMathOperator{\D}{D}
\newenvironment{bizlist}{\setcounter{Szaml}{1}
    \begin{list}{\alph{Szaml})\hfill}
    {\usecounter{Szaml}\setlength{\itemsep}{0pt}
    \setlength{\itemindent}{-\labelsep}
    \setlength{\listparindent}{0pt}}}{\end{list}}




%
% - - - -- - - S E T T I N G S ----------------
%
%\setlength{\parindent}{0pt}
%\setlength{\parskip}{\baselineskip}
\addtolength{\voffset}{-1cm}
\addtolength{\textheight}{2cm}
%\addtolength{\marginparwidth}{-1cm}
\addtolength{\hoffset}{-1cm}
\addtolength{\textwidth}{2cm}
\setlength{\headheight}{23pt}
%
\pagestyle{fancy}

  \renewcommand{\sectionmark}[1]{\markboth{\Roman{section}. tétel\\#1}{}}

\newcommand{\mktoc}{
  \pagenumbering{roman}
  \setcounter{page}{1}
  \lhead{\textbf{\thepage}}
  \cfoot{}
  \tableofcontents
  \newpage
  \lhead{\textbf{\thepage}}%/\pageref{LastPage}}
  \pagenumbering{arabic}
  \setcounter{page}{1}
}


\usepackage{booktabs}


\renewcommand{\sectionmark}[1]{\markboth{\Roman{section}. #1}{}}

\newcommand{\listazjbetu}{
  \renewcommand{\theenumi}{\alph{enumi}}
  \renewcommand{\labelenumi}{(\theenumi)}
}
\newcommand{\listazjromai}{
  \renewcommand{\theenumi}{\alph{enumi}}
  \renewcommand{\labelenumi}{(\theenumi)}
}
\newcommand{\listabetu}{
  \renewcommand{\theenumi}{\alph{enumi}}
  \renewcommand{\labelenumi}{\theenumi}
}
\newcommand{\listaszamkor}{
  \renewcommand{\theenumi}{\alph{enumi}}
  \renewcommand{\labelenumi}{\theenumi$^\circ$}
}
\newenvironment{enumzjromai}{\listazjromai\begin{enumerate}}{\end{enumerate}}
\newenvironment{enumzjbetu}{\listazjbetu\begin{enumerate}}{\end{enumerate}}

\newenvironment{enumzjr}{\begin{enumzjromai}}{\end{enumzjromai}}
\newenvironment{enumzjb}{\begin{enumzjbetu}}{\end{enumzjbetu}}


\DeclareRobustCommand{\tmspace}[3]{%
  \ifmmode\mskip#1#2\else\kern#1#3\fi\relax}
\providecommand*{\negmedspace}{\tmspace-\medmuskip{.2222em}}

\title{Visszázórendszer -- fejlesztői dokumentáció}
\author{Tóth László Attila (panther@elte.hu)}
\date{}
\begin{document}
  \maketitle
  \mktoc

  \begin{abstract}
    A visszázórendszer egy linuxra írt program, amely fájlokat küld át a hálózaton, broadcast UDP csomagokkal,
    elsősorban partíciót, vagy akár teljes merevlemez tartalmát. A nagy méretű fájlokat is kezeli, vagyis a 4 GiB-os
    határ nincs meg. A korlát 4 TiB. A kliens által letöltött adatot egyből lehet a partícióra írni. Támogatott a
    részleges áttöltés, amikor nem kell a teljes partíciót lementeni, hanem csak egy részét, amely megváltozott.
  \end{abstract}

  \section{A rendszer alapgondolata}
  Sokszor szükséges az, hogy egy-egy csupasz gépre vissza kell tölteni egy-egy nagy fájlt, egészen pontosan egy vagy
  több partíciót. Erre vannak megfelelő programok, azonban mindegyik kicsit másképp működik. Linuxon a leghasználhatóbb
  talán az \texttt{rsync} program, ez nagyon jó akkor, amikor csak kevés kliens van és a partíción található fájlokat
  kell lementeni. TCP-t használ, ezáltal ugyanannyiszor megy át a hálózaton az adat, ahány kliens csatlakozott a
  szerverhez.
  Ez a visszázórendszer többek között ezt kerüli ki: a broadcast udp használata miatt optimális esetben egy-egy csomag
  csak egyszer megy át a hálózaton. Ha a hálózaton csomagvesztés történik, akkor sincs gond, mehet tovább a rendszer,
  idővel úgyis újra el lesz küldve az elveszett csomag. Ugyanakkor nem szükséges a teljes fájl átküldése, ha a nagy
  része megegyezik, ahogy az \texttt{rsync} esetén sem kell a nem módosult állományokat újra letölteni.

  Programozói szemmel lényeges szempont a minél nagyobb hordozhatóság. A fejlécek, egyes függvények használata
  linuxhoz köti a programot, de egzotikus megoldásokat nem tartalmaz. Sokszor bináris adatokkal kell dolgozni, aminél
  számít a bájtsorrend, ezért a legegyszerűbb megoldásnak az Internet bájtsorrendjének használata bizonyult. Az STL
  nagyon sokféle típust tartalmaz, melyek nagyon jól használhatóak ebben a rendszerben is, csak olyan esetben szerepel
  alternatív megoldás a kódban, ahol a többlet funkciók miatt már nem elegendő az STL valamelyik típusa.
  
  \section{A kód (áttekintés)}
  A program kódja alapvetően három részre tagolt, a \texttt{src} könyvtáron belül található \texttt{client},
  \texttt{common}, \texttt{server} könyvtáraknak megfelelően. Ezekben a kliens által, közösen illetve a szerver által
  használt állományok találhatóak.


  \subsection{A közösen használtak}
  Itt is több részre tagolt, a \texttt{types} alatt a fájlinformációkkal és az UDP csomagok kezelésével
  kapcsolatos típusok vannak. A \texttt{rs} könyvtárban a szerver és a kliens kommunikációjához használt típusok
  szerepelnek, külön névtérben: \texttt{rs}. A használt protokoll lényegében megegyezik a \texttt{HTTP/1.0}-val, csak
  itt \texttt{RS} a neve a visszatöltő rendszer angol elnevezése alapján. Végül vannak olyan típusok, állományok, melyek
  nincsenek alkönyvtárba szervezve. Ilyenek például a konfigurációs beállításokat tartalmazó, ellenőrző összeget számító
  állományok (függvényeket tartalmaznak).
  
  \subsection{A kliens fájljai}
  A kliens viszonylag egyszerű, ezért egyetlen állományból áll: \texttt{main.cpp}, valamint a használt típusok
  implementációs állományaiból (cpp fájlok).

  \subsection{A szerver fájljai}
  A szerver többszálú, sokkal bonyolultabb, mint a kliens, így minden szál külön fájlba került. A legtöbb esetben amit a
  főprogram indít, az egy-egy függvényből áll, a kéréseket fogadó szál viszont már külön típust használ, amikor fogadta
  a kérést (hogy melyik image rész szükséges).
  
  \subsection{Tesztelések}
  A legtöbb esetben a tesztelés úgy ment, hogy ha a program kapott valamilyen adatot, akkor azt jól dolgozta-e fel,
  vagyis a kódba kiíró utasítások lettek beszúrva, melyek a végleges verzióból kikerültek.
  
  \section{A közösen használt állományok}
  
  \subsection{confparser(.cpp,.hh)}
  Ebben az állományban található a \texttt{ConfParser::ConfigParser} típus, amely beolvassa a beállításokat, és többféle
  belületet is biztosít az adatok lekérésére. Elsősorban a \texttt{std::string} használatára van kihegyezve, de a
  C-stílusú karaktertömböket is kezeli a legtöbb esetben. Ez azonban nincs használva. Így kétféleképpen lehet adatot
  lekérdezni:\\
  \texttt{std::string getValue(const std::string\& name);\\
    std::string getValueDefault(const std::string\& name, const std::string\& defaultValue);\\
  }
  Igazából mindkettőnél van alapértelmezett érték, az elsőnél az üres sztring.
  
  \subsection{md5.h, stb.}
  Ez az \texttt{md5sum} program kódjából származik, az egyetlen lényeges kód:\\
  \texttt{void *md5\_buffer (const char *buffer, size\_t len, void *digest);}\\
  Az utolsó paraméterben adja vissza a 16 bájtos eredményt.

  \subsection{rwlock}
  Ez a szerverben használt egyik legfontosabb névterület (\texttt{RWLock}). Néhány típust vezet be, az egyik
  \texttt{Lock}, amely az író-olvasó problémához ad megoldást. Itt az íróknak van elsőbbségük, azok ugyanis ritkán
  írnak.
  A zárolni kívánt szakaszt a \texttt{Reader} vagy \texttt{Writer} típussal kell védeni, a konstruktora illetve
  destruktora végzi el a szakasz zárolását, felszabadítását. Ezeket a \texttt{Lock} típus \texttt{getReader()} és
  \texttt{getWriter()} függvényeivel lehet megkapni, például:\\\\
  \texttt{\{\\
    \hspace*{5mm}RWLock::Reader r = myLock.getReader();\\
    \hspace*{5mm}\emph{kritikus szakasz}\\
    \}\\
  }

  Még egy típust vezet be, a \texttt{LockProtected\textless T\textgreater}-et amely egy template, hogy együtt lehessen
  a változó és az azt védő Lock objektum (lock és data  néven).
  
  A lock-olás megvalósítása útkifejezésekkel történik, többféleképpen is megadható lenne, itt egy bonyolultabb, de
  működő megoldás van:\\\\
  \texttt{
     \textbf{path} requestread, requestwrite \textbf{end};\\
     \textbf{path} \{ openwrite \}, openread \textbf{end};\\
     \textbf{path} \{ openread2; read \}, write \textbf{end};\\
     requestread: \textbf{begin} openread \textbf{end};\\
     openread: \textbf{begin} openread2 \textbf{end};\\
     openread2: \textbf{begin} \textbf{end};\\
     requestwrite: \textbf{begin} openwrite \textbf{end};\
     openwrite: \textbf{begin} write \textbf{end};\\
     READ: \textbf{begin} reqestread; read \textbf{end};\\
     WRITE: \textbf{begin} requestwrite \textbf{end};
  }
  
  
  \subsection{helpers}
  A \texttt{Helpers} névterület néhány függvényt és típust vezet be, ezek a következők:\\\\
  void split(const std::string\& buf,std::vector\textless std::string\textgreater\& vec);\\
  void splitBy(const char delimiter, const std::string\& buf,std::vector\textless std::string\textgreater\& vec);\\
  void splitByEx(const char delimiter, const std::string\& buf,std::vector\textless std::string\textgreater\& vec);\\\\
  std::string trimEx(const std::string\& s);\\
  std::string trimLeft(const std::string\& s);\\
  \\
  struct Digests \{\\
    char digest[16];\\
    char hexstr[33];\\
  \};
  \\
  void digest2str(Digests\& d);\\
  void str2digest(Digests\& d);\\
  \\
  template \textless class T \textgreater\\
  T max(T a, T b) \{\\
  \hspace*{2mm} return a \textgreater b ? a : b;\\
  \}\\
  \\
  template \textless class T\textgreater\\
  T min(T a, T b) \{\\
  \hspace*{2mm}  return a \textless b ? a : b;\\\\
  \}

  Ezek rendre a következőt tudják a \texttt{split*} a \texttt{buf} sztringet vágja szét a \texttt{vec} tömbbe. A
  \texttt{split} a szóköz, a \texttt{splitBy} a \texttt{delimiter} mentén, a \texttt{splitByEx} szintén, csak itt két
  részre vágja, nem minden \texttt{delimiter} mentén vág.
  
  A \texttt{Digest} kényelmi okokból bevezetett típus, így egyszerűbb lett a következő két függvény használata, mely md5
  digest (16 karakter) és az azt reprezentáló hexadecimális számokból álló sztring között konvertál.

  Néha szükséges a \texttt{min} és a \texttt{max} használata, ezért ezt is megtalálható itt, az általánosság célja miatt
  sablonként.
  
  A \texttt{trimEx} lényegében ugyanazt csinálja, mint a \texttt{trim} tenné, csak éppen nem a jobb feléről vágja le a
  szóközt, hanem az adott szó utáni első szóköztől vág. A \texttt{trimLeft} csak a baloldali szóközöket eszi meg.
  

  \subsection{checksum}
  Itt a következő függvények szerepelnek, öndokumentálva, ahol a programban az utolsó használt csak:\\\\
  \texttt{
    /// calculates checksume of a part of a file. Return value indicates success\\
    bool calculateChecksum(int fd, off64\_t offset, size\_t maxLength, ChecksumInfo* ci, size\_t\& readLength);\\
    /// calculates every blocks's checksum, every block's size is \textless= block\_size\\
    bool calculateChecksum(int fd, ChecksumInfoList* checksums, size\_t block\_size);\\
    /// calculates every block's checksum, block size is 1024*1024 bytes\\
    bool calculateChecksum(int fd, ChecksumInfoList* checksums);\\\\
  }

  Mindegyik egy-egy megnyitott fájlleírót használ, és a típusoknál tárgyalt \texttt{ChecksumInfoList}-be ment.

  \subsection{iostream}
  Ez a fejléc az iostream header include-olja és bevezeti a \texttt{cin}, \texttt{cerr}, \texttt{cout}, \texttt{endl}
  neveket a globális névtérbe.
  
  \subsection{types/bitvect, types/bitvectex}
  A \texttt{BitVector} és a \texttt{BitVectorEx} típusokat tartalmazzák, a különbség csak annyi, hogy az utóbbi kiír
  pontokat, ha olyan bitet kell beállítani, amelynek az előzője lett előtte igazra állítva, valamint kiír egy
  \texttt{|szám|} alakú részt, ahol \texttt{szám} darab bit nem lett igazra állítva.
  
  A felületük teljesen azonos:\\\\
  \texttt{
    void set(const int\& index );\\
    void unset(const int\& index );\\
    bool operator[](const int\& index) const;\\
    void intv(int\& from, int\&  to) const;\\
    void intv1(int\& from, int\&  to) const;\\\\
  }
  A \texttt{set} és a \texttt{unset} állítja egyesre illetve nullára az adott bitet, az \texttt{operator[]()} egy adott
  bit értékét adja vissza, végül a \texttt{intv} és \texttt{intv1} a paraméterben megadott változókat állítja be
  azt jelezve, hogy az elejétől kezdve mettől meddig van csupa $0$-ból vagy csupa egyesből álló rész. Amennyiben nincsen
  ilyen, akkor a második paraméter, a \texttt{to} értéke $-1$ lesz.

  Fontos még a konstruktor is, mivel a bitvector nem átméretezhető, vagyis a konstruktorparaméterben megadott egész szám
  lesz a mérete az objektum teljes élettartama alatt.

  \subsection{types/checksuminfo}
  Ebben kétféle típus szerepel, a \texttt{ChecksumInfo} és a \texttt{ChecksumInfoList}. Az utóbbi egy wrapper az
  \texttt{std::vector}-ra. Az előbbinek egyetlen adattagja van:\\
  \texttt{char digest[16];}

  Mindkettőre igaz az, hogy a copy és alapértelmezett constructor, értékadó operátor, összehasonlítő műveletek és a
  ki- és bemeneti operátorok definiáltak.

  A szokásostól eltérően az \texttt{operator\textless\textless()} és \texttt{operator\textgreater\textgreater()}
  operátorok binárisan írják és olvassák az adatot, azaz az egész számok is binárisan reprezentáltak. Ezért fontos a
  bájt sorrend, ennek megoldására a a \texttt{htonl} és hasonló függvények használata szolgált. Vagyis az Internet
  bájtsorrendjében tárolódik minden, ugyanez van használva a hálózati kommunikáció esetén is.

  A \texttt{ChecksumInfoList} még nem említett felülete:\\\\
  \texttt{
    size\_t size() const;\\
    void add(ChecksumInfo\& info);\\
    ChecksumInfo operator[](int idx) const;\\
    void clear();
  }
  
  \subsection{types/imageinfo}
  Ez az előzővel lényegében teljesen azonos módon működik, az eltérés csak a \texttt{ImageInfo} adattagjaiban van:
  \texttt{
    std::string name;\\
    char  digest[16];\\
    int64\_t size;\\
    int32\_t blockSize, lastBlockSize;\\
    time\_t lastModified;    
  }

  A \texttt{name} a szerver konfigfájljában szereplő név, a \texttt{digest} ennek md5 kivonata, a \texttt{size} pedig a
  hozzá tartozó fájl mérete. A \texttt{blockSize} fixen 1 MiB, de a \texttt{lastBlockSize} nem, mivel ritka az olyan
  állomány, amely pontosan az 1 MiB többszöröse lenne.
  
  \subsection{types/semaphore}
  A \texttt{Semaphore} típussal implementáltak a fentebb említett útkifejezések. A kód a \texttt{CommonC++}
  módosításával készült, az abban megadott kommentek magyarázzák, mit csinálnak a függvények:\\\\\\
  \texttt{
    typedef unsigned long   timeout\_t;\\
    class SemaphoreException \{\};\\
     \\
     class Semaphore\\
    \{\\\\
    public:\\\\
    /**\\
     * The initial value of the semaphore can be specified.  An initial\\
     * value is often used When used to lock a finite resource or to\\
     * specify the maximum number of thread instances that can access a\\
     * specified resource.\\
     *\\
     * @param resource specify initial resource count or 1 default.\\
     */\\
    Semaphore(unsigned resource = 1);\\
    \\
    /**\\
     * Destroying a semaphore also removes any system resources\\
     * associated with it.  If a semaphore has threads currently waiting\\
     * on it, those threads will all continue when a semaphore is\\
     * destroyed.\\
     */\\
    virtual \~{}Semaphore();\\
    \\
    /**\\
     * Wait is used to keep a thread held until the semaphore counter\\
     * is greater than 0.  If the current thread is held, then another\\
     * thread must increment the semaphore.  Once the thread is accepted,\\
     * the semaphore is automatically decremented, and the thread\\
     * continues execution.\\
     *\\
     * The pthread semaphore object does not support a timed "wait", and\\
     * hence to maintain consistancy, neither the posix nor win32 source\\
     * trees support "timed" semaphore objects.\\
     *\\
     * @return false if timed out\\
     * @param timeout period in milliseconds to wait\\
     * @see \#post\\
     */\\
    bool wait(timeout\_t timeout = 0);\\
    \\
    /// Ugyanaz, ahogy Dijkstra használta:\\
    void P() \{ wait(); \}\\
    \\
    \\
    /**\\
     * Posting to a semaphore increments its current value and releases\\
     * the first thread waiting for the semaphore if it is currently at\\
     * 0.  Interestingly, there is no support to increment a semaphore by\\
     * any value greater than 1 to release multiple waiting threads in\\
     * either pthread or the win32 API.  Hence, if one wants to release\\
     * a semaphore to enable multiple threads to execute, one must perform\\
     * multiple post operations.\\
     *\\
     * @see \#wait\\
     */
    void post(void)\\;
    \\
    /// Ugyanaz, ahogy Dijkstra használta
    void V() \{ post(); \};\\
    \\
    // FIXME: how implement getValue for posix compatibility ?\\
    /**\\
     * Get the current value of a semaphore.\\
     *\\
     * @return current value.\\
     */
     int getValue(void);\\
    \};
     \\
     /**
     * The SemaphoreLock class is used to protect a section of code through\\
     * a semaphore so that only x instances of the member function may\\
     * execute concurrently.\\
     *\\
     * A common use is\\
     *\\
     * void func\_to\_protect()\\
     * \{\\
     *   SemaphoreLock lock(semaphore);\\
     *   ... operation ...\\
     * \}\\
     *\\
     * NOTE: do not declare variable as "SemaohoreLock (semaphore)", the\\
     * mutex will be released at statement end.\\
     *\\
     * @author David Sugar \textless dyfet@gnu.org\textgreater\\
     * @short Semaphore automatic locker for protected access.\\
     */\\
     class SemaphoreLock\\
     \{\\
     public:\\\\
     /**\\
     * Wait for the semaphore\\
     */\\
     SemaphoreLock( Semaphore\& \_sem );\\\\
     /**\\
     * Post the semaphore automatically\\
     */\\
     // this should be not-virtual\\
     \~{}SemaphoreLock();\\
     \};
    }      


  Az RWLock::Lock az utóbbi típus alapján készült.

  \subsection{types/udpdata}
  Az \texttt{UDPData} típus felülete:\\\\
  \texttt{
    char digest[16]; // md5sum of imgname\\
    int32\_t index;\\
    int32\_t checksum;\\
    char data[1024];\\\\
  }
  Ez az, ami a broadcast csomagokban utazik. Az \text{index} jelzi, hogy hanyadik 1024 bájtos blokkja a fájlnak. A
  \texttt{checksum} pedig a \texttt{data} rész karaktereinek összeadásából származik.
  
  \subsection{rs/rsparser}
  Az \texttt{RS::Parser} típus a hálózati foglalaton érkező adatot (fejlécet) értelmezi, és soronként bemásolja egy
  \texttt{std::vectorba}. Ha valahol hibát észlel, azt jelzi. Az értelmező függvény:\\

  \texttt{int parseHeader(int\& fd, fd\_set *readfds, fd\_set *excepfds);}\\
  
  Ez a függvény arra lett kitalálva, hogy az objektumot használó függvény \texttt{select} utasításában meghívja, ha
  érkezhet adat. A hiba \texttt{fd\_set} csak az OOB (Out Of Bound) adat kezelésére van.
  
  Ha hiba van a folyamban (foglalaton), akkor azt lezárja és -1 értéket ad neki. Ha minden műkdödik, akkor addig olvassa
  a fejlécet, amíg egy üres sort nem talál. Mivel ekkor fennmaradhat adat, ezért szükség lehet annak lekédezésére:\\
  \texttt{ std::string getRemainingData() const;}

  \subsection{rs/rsheader}
  A válaszüzenethez használt \texttt{RS::Header} típus, felülete:\\\\
  \texttt{
  void setHeader(const std::string\& name, const std::string\& value);\\
  const std::string getLines() const;\\
  std::string getHeader(const std::string\& name) const {return getHeader(name, ""); }\\
  std::string getHeader(const std::string\& name, const std::string\& defaultValue) const;\\
  }
  
  Az utóbbi kettő függvény fölösleges, kivéve egységteszteléshez. A típus védett adattagja a \texttt{std::string
  firstLine}, mivel az nem triviális, hogy hogyan néz ki - a származtatott osztály állítja be. A \texttt{getLines()}
  adja vissza a teljes generált fejlécet.
  
  \subsection{rs/rsserverresponse}
  A \texttt{RS::Header} osztályból származó \texttt{RS::ServerResponse} típus, amely az első sort állítja be a
  következő függvények révén:\\\\
  \texttt{
    void setVersion(const std::string\& version);\\
    void setStatus(const std::string\& status);\\
    void setStatusText(const std::string\& statusText);\\
  }
  
  Ezek pontosan ugyanúgy viselkednek, mint a \texttt{HTTP} protokoll esetén.
       
  A szerver ezzel válaszol a kliens valamilyen kérésére.
  
  \subsection{rs/clientrequest}
  A \texttt{RS::ClientRequest} a \texttt{RS::ServerResponse}-hoz hasonló típus, a \texttt{RS::Header} osztályból
  származik és a következő függvényeket definiálja:\\\\
  \texttt{
    void setVersion(const std::string\& version);\\
    void setImageName(const std::string\& imageName);\\
  }
  
  A HTTP-től eltérés az utóbbi, de igazából csak a név más, ugyanis a HTTP estén ez az URL lenne.
  
  A kliens ezt küldi el.
  
  \subsection{rs/rsserverrequest}
  Az \texttt{RS::ServerRequest} önálló típus lett, bár szinte teljesen megegyezik az \texttt{RS::ClientResponse}
  osztállyal. A szerver ezt olvassa, mint a klienstől kapott kérést.

  Felülete:\\\\
  \texttt{
    ServerRequest(const RS::Parser\& parser);\\
    ServerRequest(const RS::Parser::strList\& lines);\\
    std::string getMethod() const;\\
    std::string getURL() const;\\
    std::string getVersion() const;\\
    std::string getHeader(const std::string\& name) const {return getHeader(name, ""); }\\
    std::string getHeader(const std::string\& name, const std::string\& defaultValue) const;\\
    int getError()  const;\\
    bool hasError() const;\\
  }

  A \texttt{getMethod()} nem használt, szerepe a HTTP-szerű \texttt{GET}, esetleg \texttt{POST} megkülönböztetése lett
  volna, de végül elég lett a GET használata is.
  
  A \texttt{getURL()} az elküldött image nevét adja vissza.

  A kérésben lehet hiba is, ekkor a \texttt{getError()} a hibás sor számát (0-val kezdve) adja vissza, valamint -1-et ha
  nem történt hiba.
  
  Az \texttt{RS::Parser::strList} egy typedef a \texttt{std::vector\textless std::string\textgreater} típusra.
  \subsection{rs/rsclientresponse}
  Az \texttt{RS::ClientResponse} az \texttt{RS::ServerRequest} osztállyal. A szervertől kapott válasz a kliensprogramban.
  
  Felülete:\\
  \texttt{
    ClientResponse(const RS::Parser\& parser);\\
    ClientResponse(const RS::Parser::strList\& lines);\\
    std::string getStatus() const;\\
    std::string getStatusText() const;\\
    std::string getVersion() const;\\
    std::string getHeader(const std::string\& name) const {return getHeader(name, ""); }\\
    std::string getHeader(const std::string\& name, const std::string\& defaultValue) const;\\
    int getError() const;\\
    bool hasError() const;\\
  }

  Látható, hogy szinkronban van a \texttt{RS::ServerResponse} felületével.
  
  \newpage
  \section{A szerver}
  A szerver a \texttt{pthread} függvénykönyvtárat használja a többszálúság kezelésére.

  Az elindulás, opciók feldolgozása után azonnal értelmezi a konfigurációs állományt, hiba esetén kilép. Ha sikeres volt
  a beolvasás, és ezután az image lista is elkészült, akkor elindítja a szálakat. Ezek:
  \begin{itemize}
    \item info thread, amely egy-egy image-ről ad információt a kliensnek, vagyis az ellenőrző összegeket, méretet, stb.
    \item config thread, amely 30 percenként frissíti a beállításokat, valamint a hozzá tartozó ImageInfoList típusú
    objektumot.
    \item server thread, amely a kéréseket fogadja, majd indít egy-egy újabb szálat minden egyes kéréshez
    \item restore thread, amely az udp broadcast üzeneteket küldi
  \end{itemize}
  
  \subsection{config(.cpp,.hh)}
  Itt találhatóak a konfigurációs állomány és az image lista feldolgozásával kapcsolatos függvények. Ezek ellenőrzik a
  config file tartalmát, az image listát frissítik, mivel ki lehet menteni egy fájlba, amit a config fájl
  \texttt{cache-directory} beállításának megfelelő könyvtárba, \texttt{imageinfo} néven ment el. Ha létezik, akkor
  visszatölti, és mindig ellenőrzi, hogy az adott fájl változott-e (méret és módosítási idő alapján). Ha nem, akkor
  békén hagyja, ha igen, akkor újraszámolja az ellenőrző összegeket. Ha ezzel kész van, akkor kimenti a fentebb említett
  állományba.

  Alapvetően egyetlen fontos függvénye van, aminek publikusnak kell lenni (vagyis névtelen namespace-ben nem lehet):\\

  \texttt{bool parseConfig(const std::string\& cfgfile);}\\

  Ahol a cfgfile paramétert a main fv adja először. Hamissal tér vissza, ha sikertelen volt valami, illetve kilépteti a
  programot, ha nem volt korábbi konfigurációs beállítás.
  
  Ha valamelyik beállítás rossz, akkor kiírja/megjelöli, hol a hiba
  
  A \texttt{config.hh} bevezeti a következő típust:\\\\
  typedef struct Configuration \{\\
  \hspace*{5mm} ConfParser::ConfigParser parser;\\
  \hspace*{5mm}ImageInfoList infoList;\\
  \} Configuration;\\

  Az ebből származó ``konfig változó''\\
  
  \texttt{RWLock::LockProtected\textless Configuration*\textgreater  configuration;},\\\\
  ami az előfordítói utasításoktól függően lehet \texttt{extern}, vagy nem az, vagy egyáltalán nem jelenik meg abban az
  állományban, amely ezt a fejlécet beillesztette.

  \newpage
  \subsection{request.hh}
  Ez is bevezet egy védett változót, \texttt{requests} néven, ahol a \texttt{data} adattag ``felülete'':\\\\
  \texttt{
    std::string name;\\
    int from;\\
    int to;\\
    int max;\\
  }

  Itt a \texttt{name} az image neve, \texttt{from}, \texttt{to} az, hogy melyik 1024 bájtos blokkokból álló
  intervallumra van szükség, a \texttt{max} pedig azt jelzi, hogy melyik az utolsó szóbajöhető blokk (ezutáni már nincs
  a fájlban).

  \subsection{requestthread}
  Ez egy típus, \texttt{RequestThread}, amely kicsit hasonlít a szálakat mint típusokat definiáló osztályokra:
  \texttt{void run()} függvényével implementálja a szál kódját.
  
  Beolvassa az elküldött fejlécet, ha az jó, akkor feldolgozza, de ha nem lezárt a foglalat, akkor előtte elküld egy
  választ. Itt a következő válaszüzenetek lehetnek, ha hiba történt, 400-as státusszal (egyébként 200, mint a HTTP
  esetén is).
  
  \begin{itemize}
  \item "Image not found"
  \item "Image found but digest doesn't match"
  \item "Value of 'from' or 'to' is out of bound"\\
    vagyis valamelyik érvénytelen (negatív), vagy éppen a from nagyobb, mint a to.
  \item  "Value of 'from' is out of bound"
  \item  "Value of 'to' is out of bound");
  \item  "Request is invalid, see line no \textless szám \textgreater" 
  \end{itemize}
  
  A szál a végén, ha már semmi hiba nem votl, írásra zárolja a requests változót és hozzáfűzi az kérést.
  
  A kérés formája:\\\\
  \texttt{
    GET img0 RS/1.0\textbackslash r\textbackslash n\\
    digest: \textless digest\textgreater\textbackslash r\textbackslash n\\
    from: 0\textbackslash r\textbackslash n\\
    to: 2\textbackslash r\textbackslash n\\
    \textbackslash r\textbackslash n\\\\
  }
  ahol  digest nem kötelező.
  
  \subsection{restorethread}
  Ez is egy osztály, a \texttt{run()} függvényében küldi el a broadcast csoamgokat.
  
  A kód lényegében egy ciklus, ahol az elején végignézi az összes kérést, ez alapján a függvényben tárolt tömbök
  segítségével számon tartja, hogy milyen kérések érkeztek. Mivel \texttt{BitVector}-t használ annak ellenőrzésére, hogy
  milyen részekre van igény, ezért mindig az elejéről kezdi el küldeni a csomagokat. Ez különleges esetben
  kiéheztetéshez vezet, ugyanis előfordulhat, mindig csak az elejéről küld adatot, sosem ér a végére, hiszen az elejéről
  kér valaki újra és újra.

  Minden szóba jöhető image-et végignéz, mindig csak egyet-egyet, s amelyikből van igény, abból küld. Ez megint nem
  hatékony, ha már sok különbözőt kértek, mert így egyre ritkábban kerül sorra ugyanaz az image. A jelenlegi
  megvalósítás célja az átlátható, egyszerű kód volt.

  Egyszerre legfeljebb 1000 csomagot küld ki, minden 60. után áll egy mikrosekundumot.

  Az \texttt{UDPData}-ban szereplő egészek is a hálózat bájt sorrendjének megfelelően utaznak.
  
  \subsection{server\_thread}
  A \texttt{server\_thread} függvény a kéréseket fogadó szál kódja. Amint jön egy kérés, egy \texttt{RequestThread} objektum jön
  létre, amely feldolgozza a kérést.
  
  \newpage
  \section{A kliens}
  A korábban említett módon viselkedik, a logika függvényekre tagolt. A program nagy része a \texttt{main()} függvény
  része, de a segítség kiírása, adott image adatainak lekérdezése illetve image részérre vonatkozó kérés küldése külön
  függvénybe került, hogy a kód ne legyen átláthatatlan.
  
  A kliens a szervertől kapott adatok ismeretében ellenőrzi, hogy mennyi van meg az adott fájlból. Ehhez az ellenőrző
  összegek ismerete kell, ha megegyezik valamelyik darab, akkor az annak megfelelő részt letöltöttre állítja be a
  BitVectorEx típusú objektumban. 

  A fő ciklus elején egy \texttt{select()} hívás van, ami alapján eldől, hogy fogadhat-e adatot a kliens, vagy inkább a
  szervertől kérjen még egy darabot.
  
  A \texttt{-d} opcióval részletesen kiírja, mikor mit csinál, ellenkező esetben valami olyasmi fog szerepelni a
  képernyőn, mint az alábbi (rövidített) sor:\\\\  
  \texttt{......|8201|............|-8279|....}\\
  
  Itt az látszik, hogy kezdetben egymás után érkeznek a csomagok, amik a fájlban is egymás után vannak, azonban utána
  sokat ugrik a számláló, és jóval arrébb található részt fogad a kliens. Ezután megint egy ugrás jön, és a korábban
  kimaradt rész érkezik.
  
  \newpage
  \section{Tesztelés}
  Elsősorban a szerver esetén bizonyos komponenseket lehetett önállóan tesztelni, ilyen az \texttt{RWLock::Lock}
  használata, vagy épp a típusok ellenőrzése. A \texttt{BitVector} esetén elég volt beleírni a tagfüggvények kódjába
  kiíró utasításokat, de az \texttt{ImageInfo}, \texttt{RS::Parser} osztályok komplexek. A kérések feldolgozása is hibaforrás.

  \subsection{A kérések ellenőrzése}
  A kéréseket (\texttt{requests} változó) a következő, módosított függvénnyel lehetett ellenőrzni:
  \texttt{
    void RestoreThread::run()\\
    \{\\
    \hspace*{4mm}for (int idx = 0; ; ++idx) \{\\
    \hspace*{8mm}sleep (5);\\
    \hspace*{8mm}\{\\
    \hspace*{12mm}RWLock::Reader r = ::requests.lock.getReader();\\
    \hspace*{12mm}int i =0;\\
    \hspace*{12mm}std::vector\textless RequestInfo\textgreater::const\_iterator it = requests.data.begin();\\
    \hspace*{12mm}for (;it != requests.data.end(); ++it, ++i) \{\\
    \hspace*{12mm}cout  \textless\textless  idx \textless \textless ": "   \textless\textless i \textless\textless ": "
    \textless\textless (*it).name \textless\textless " "\\
    \hspace*{20mm}\textless\textless it-\textgreater from \textless\textless " "
    \textless\textless it-\textgreater to  \textless\textless endl;\\
    \hspace*{8mm}\}\\
    \hspace*{4mm}\}\\
    \}    
  }
  

\end{document}

% Local Variables:
% fill-column: 120
% TeX-master: t
% End:
