\documentclass[fleqn,10pt,a4paper,titlepage]{article}
\includeonly{common,komplexfv}

%
% - ---- -- PACKAGES--------------------------
%

\usepackage{amssymb}
\usepackage{amsmath}
\usepackage[utf8]{inputenc}
\usepackage[magyar]{babel}
%\usepackage{amsthm}
\usepackage{t1enc}
\usepackage{theorem}
\usepackage{fancyhdr}
\usepackage{lastpage}
\usepackage{paralist}

%
% ---------- CODES --------------------------
%
\makeatletter
\gdef\th@magyar{\normalfont\slshape
  \def\@begintheorem##1##2{%
  \item[\hskip\labelsep \theorem@headerfont ##2.\ ##1.]}%
  \def\@opargbegintheorem##1##2##3{%
  \item[\hskip\labelsep \theorem@headerfont ##2. ##1.\ (##3)]}}
\makeatother


%
% ------------  N E W  C O M M A N D S --------
%
%\newcommand{\ob}{\begin{flushright} \leavevmode\hbox to.77778em{\hfil\vrule
%    \vbox to.675em{\hrule width.6em\vfil\hrule}\vrule}\hfil\end{flushright}}
\newcommand{\ob}{\hfill$\square$}
\newcommand{\ff}{f\in\mathbb{R}\rightarrow\mathbb{R}}
\newcommand{\fab}{f\colon (a,b)\rightarrow\mathbb{R}}
\newcommand{\fabk}{f\colon \left[a,b\right]\rightarrow\mathbb{R}}
\newcommand{\fir}{f\colon I\rightarrow\mathbb{R}}
\newcommand{\fdab}{f\in D(a,b)}
\newcommand{\fcab}{f\in C[a,b]}
\newcommand{\exist}{\exists}
\newcommand{\ek}{\Longleftrightarrow}
\newcommand{\la}{\lambda}
\newcommand{\ro}{\varrho}
\newcommand{\K}{\ensuremath{\mathbb{K}}}
\newcommand{\R}{\ensuremath{\mathbb{R}}}
\newcommand{\Q}{\ensuremath{\mathbb{Q}}}
\newcommand{\N}{\ensuremath{\mathbb{N}}}
\newcommand{\C}{\ensuremath{\mathbb{C}}}
\newcommand{\n}{\ensuremath{\to}} %azonos a rightarrow-val
%duplanyil, szuksegesseg
\newcommand{\nn}{\ensuremath{\Rightarrow}}
%\newcommand{\Omage}{\Omega}
%elegségesség, nem def :)
%\newcommand{\nb}{\Leftarrow}
\newcommand{\di}{\displaystyle}
\newcommand{\sarrow}{\downarrow}
\newcommand{\narrow}{\uparrow}
\newcommand{\lt}{<}
\newcommand{\gt}{>}
\newcommand{\Int}{\intop\limits}
\newcommand{\ures}{\varnothing}
\newcommand{\ekv}{\iff}
\newcommand{\ekviv}{\ekv}
\renewcommand{\epsilon}{\varepsilon}
\newcommand{\eps}{\varepsilon}
%
% ------------  NEW PART DEFS -----------------
%
\newcounter{Szaml}


\theoremstyle{magyar}
\theoremheaderfont{\itshape\bfseries}
\newtheorem{de}{definíció}[section]
\newtheorem{te}{tétel}[section]
\newtheorem{bi}{bizonyítás}[section]
\newtheorem{ko}{következmény}[section]
\newtheorem{me}{megjegyzés}[section]
\newtheorem{al}{állítás}[section]


\newenvironment{korlista}{\begin{enumerate}[\quad1$^\circ$]}{\end{enumerate}}

\newenvironment{biz}{\begin{trivlist}\item\relax\mbox{\textbf{Bizonyítás.\enskip}}\ignorespaces}{\ob\end{trivlist}}
\newenvironment{Biz}{\begin{trivlist}\item\relax\mbox{\textbf{Bizonyítás.\enskip}}\ignorespaces\begin{korlista}}{\ob\end{korlista}\end{trivlist}}
\newenvironment{kov}{\begin{trivlist}\item\relax\mbox{\textbf{Következmény.\enskip}}\ignorespaces}{\end{trivlist}}
\newenvironment{megj}{\begin{trivlist}\item\relax\mbox{\textbf{Megjegyzés.\enskip}}\ignorespaces}{\end{trivlist}}
\newenvironment{Megj}{\begin{megj}\begin{korlista}}{\end{korlista}\end{megj}}
\newenvironment{pl}{\begin{trivlist}\item\relax\mbox{\textbf{Példa.\enskip}}\ignorespaces}{\end{trivlist}}
\newenvironment{Pl}{\begin{pl}\begin{korlista}}{\end{korlista}\end{pl}}
\DeclareMathOperator{\D}{D}
\newenvironment{bizlist}{\setcounter{Szaml}{1}
    \begin{list}{\alph{Szaml})\hfill}
    {\usecounter{Szaml}\setlength{\itemsep}{0pt}
    \setlength{\itemindent}{-\labelsep}
    \setlength{\listparindent}{0pt}}}{\end{list}}




%
% - - - -- - - S E T T I N G S ----------------
%
%\setlength{\parindent}{0pt}
%\setlength{\parskip}{\baselineskip}
\addtolength{\voffset}{-1cm}
\addtolength{\textheight}{2cm}
%\addtolength{\marginparwidth}{-1cm}
\addtolength{\hoffset}{-1cm}
\addtolength{\textwidth}{2cm}
\setlength{\headheight}{23pt}
%
\pagestyle{fancy}

  \renewcommand{\sectionmark}[1]{\markboth{\Roman{section}. tétel\\#1}{}}

\newcommand{\mktoc}{
  \pagenumbering{roman}
  \setcounter{page}{1}
  \lhead{\textbf{\thepage}}
  \cfoot{}
  \tableofcontents
  \newpage
  \lhead{\textbf{\thepage}}%/\pageref{LastPage}}
  \pagenumbering{arabic}
  \setcounter{page}{1}
}


\usepackage{booktabs}


\renewcommand{\sectionmark}[1]{\markboth{\Roman{section}. #1}{}}

\newcommand{\listazjbetu}{
  \renewcommand{\theenumi}{\alph{enumi}}
  \renewcommand{\labelenumi}{(\theenumi)}
}
\newcommand{\listazjromai}{
  \renewcommand{\theenumi}{\alph{enumi}}
  \renewcommand{\labelenumi}{(\theenumi)}
}
\newcommand{\listabetu}{
  \renewcommand{\theenumi}{\alph{enumi}}
  \renewcommand{\labelenumi}{\theenumi}
}
\newcommand{\listaszamkor}{
  \renewcommand{\theenumi}{\alph{enumi}}
  \renewcommand{\labelenumi}{\theenumi$^\circ$}
}
\newenvironment{enumzjromai}{\listazjromai\begin{enumerate}}{\end{enumerate}}
\newenvironment{enumzjbetu}{\listazjbetu\begin{enumerate}}{\end{enumerate}}

\newenvironment{enumzjr}{\begin{enumzjromai}}{\end{enumzjromai}}
\newenvironment{enumzjb}{\begin{enumzjbetu}}{\end{enumzjbetu}}


\DeclareRobustCommand{\tmspace}[3]{%
  \ifmmode\mskip#1#2\else\kern#1#3\fi\relax}
\providecommand*{\negmedspace}{\tmspace-\medmuskip{.2222em}}

\title{Visszázórendszer felhasználói dokumentáció)}
\author{Tóth László Attila (panther@elte.hu)}
\date{}
\begin{document}
  \maketitle
  \mktoc

  \begin{abstract}
    A visszázórendszer egy linuxra írt program, amely néhány fájlt küld át a hálózaton, broadcast UDP csomagokkal. A
    kliens ezt egy fájba menti, mely akár lehet egy speciális eszközfájl is, ezáltal egy merevlemez vagy partíció
    közvetlenül visszatölthető. Ugyanakkor nem feltétlenül az egészet fájlt, mivel mind a szerver, mind a kliens
    ellenőrző összegeket számol, csak az eltérő részek utaznak a hálózaton. A 4GiB-nál nagyobb fájlokat is kezeli a
    rendszer.
    
    Akkor igazán hatékony, ha sok kliens fut egyszerre, így egyszerre több számítógépen lehet a visszatöltést elvégezni.
  \end{abstract}

  \section{A rendszer alapgondolata}
  Sokszor szükséges az, hogy egy-egy csupasz gépre vissza kell tölteni egy-egy nagy fájlt, egészen pontosan egy vagy
  több partíciót. Erre vannak megfelelő programok, azonban mindegyik kicsit másképp működik. Linuxon a leghasználhatóbb
  talán az \texttt{rsync} program, ez nagyon jó akkor, amikor csak kevés kliens van és a partíción található fájlokat
  kell lementeni. TCP-t használ, ezáltal ugyanannyiszor megy át a hálózaton az adat, ahány kliens csatlakozott a
  szerverhez.
  Ez a visszázórendszer többek között ezt kerüli ki: a broadcast udp használata miatt optimális esetben egy-egy csomag
  csak egyszer megy át a hálózaton. Ha a hálózaton csomagvesztés történik, akkor sincs gond, mehet tovább a rendszer,
  idővel úgyis újra el lesz küldve az elveszett csomag. Ugyanakkor nem szükséges a teljes fájl átküldéese, ha a nagy
  része megegyezik, ahogy az \texttt{rsync} esetén sem kell a nem módosult fájlokat újra letölteni.
  
  \section{A program fordítása, indítása}
  A program gcc 4.0.3-as fordítóval tesztelt, régebbivel is le kell fordulnia (pl. 3.4.x). A forrás az \texttt{src}
  könyvtárban található, oda belépve egy \texttt{make} parancsot kell beírni a parancssorba. A program többszálú, ezért
  a \texttt{pthread} függvényeket használja.

  Az indításról szóló rész, opciókkal, mindennel megtalálható a felhasználói leírásban.

  \section{A kód (áttekintés)}
  A program kódja alapvetően három részre tagolt, a \textt{src} könyvtáron belül található \texttt{client},
  \texttt{common}, \texttt{server} könyvtáraknak megfelelően. Ezekben a kliens által, közösen illetve a szerver által
  használt állományok találhatóak.

  \subsection{A közösen használtak}
  Itt is több részre tagolt, a \texttt{types} alatt a fájlinformációkkal és az UDP csomagok kezelésével
  kapcsolatos típusok vannak. A \texttt{rs} könyvtárban a szerver és a kliens kommunikációjához használt típusok
  szerepelnek, külön névtérben: \texttt{rs}. A használt protokoll lényegében megegyezik a \texttt{HTTP/1.0}-val, csak
  itt \textt{RS} a neve a visszatöltő rendszer angol elnevezése alapján. Végül vannak olyan típusok, állományok, melyek
  nincsenek alkönyvtárba szervezve. Ilyenek például a konfigurációs beállításokat tartalmazó, ellenőrző összeget számító
  állományok (függvényeket tartalmaznak).
  
  \subsection{A kliens fájljai}
  A kliens viszonylag egyszerű, ezért egyetlen állományból áll: \texttt{main.cpp}, valamint a használt típusok
  implementációs állományaiból (cpp fájlok).

  \subsection{A szerver fájljai}
  A szerver többszálú, sokkal bonyolultabb, mint a kliens, így minden szál külön fájlba került. A legtöbb esetben amit a
  főprogram indít, az egy-egy függvényből áll, a kéréseket fogadó szál viszont már külön típust használ, amikor fogadta
  a kérést (hogy melyik image rész szükséges).
  
  
  
  

\end{document}

% Local Variables:
% fill-column: 120
% TeX-master: t
% End:
