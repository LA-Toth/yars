\documentclass[fleqn,10pt,a4paper,titlepage]{article}
\includeonly{common,komplexfv}

%
% - ---- -- PACKAGES--------------------------
%

\usepackage{amssymb}
\usepackage{amsmath}
\usepackage[utf8]{inputenc}
\usepackage[magyar]{babel}
%\usepackage{amsthm}
\usepackage{t1enc}
\usepackage{theorem}
\usepackage{fancyhdr}
\usepackage{lastpage}
\usepackage{paralist}

%
% ---------- CODES --------------------------
%
\makeatletter
\gdef\th@magyar{\normalfont\slshape
  \def\@begintheorem##1##2{%
  \item[\hskip\labelsep \theorem@headerfont ##2.\ ##1.]}%
  \def\@opargbegintheorem##1##2##3{%
  \item[\hskip\labelsep \theorem@headerfont ##2. ##1.\ (##3)]}}
\makeatother


%
% ------------  N E W  C O M M A N D S --------
%
%\newcommand{\ob}{\begin{flushright} \leavevmode\hbox to.77778em{\hfil\vrule
%    \vbox to.675em{\hrule width.6em\vfil\hrule}\vrule}\hfil\end{flushright}}
\newcommand{\ob}{\hfill$\square$}
\newcommand{\ff}{f\in\mathbb{R}\rightarrow\mathbb{R}}
\newcommand{\fab}{f\colon (a,b)\rightarrow\mathbb{R}}
\newcommand{\fabk}{f\colon \left[a,b\right]\rightarrow\mathbb{R}}
\newcommand{\fir}{f\colon I\rightarrow\mathbb{R}}
\newcommand{\fdab}{f\in D(a,b)}
\newcommand{\fcab}{f\in C[a,b]}
\newcommand{\exist}{\exists}
\newcommand{\ek}{\Longleftrightarrow}
\newcommand{\la}{\lambda}
\newcommand{\ro}{\varrho}
\newcommand{\K}{\ensuremath{\mathbb{K}}}
\newcommand{\R}{\ensuremath{\mathbb{R}}}
\newcommand{\Q}{\ensuremath{\mathbb{Q}}}
\newcommand{\N}{\ensuremath{\mathbb{N}}}
\newcommand{\C}{\ensuremath{\mathbb{C}}}
\newcommand{\n}{\ensuremath{\to}} %azonos a rightarrow-val
%duplanyil, szuksegesseg
\newcommand{\nn}{\ensuremath{\Rightarrow}}
%\newcommand{\Omage}{\Omega}
%elegségesség, nem def :)
%\newcommand{\nb}{\Leftarrow}
\newcommand{\di}{\displaystyle}
\newcommand{\sarrow}{\downarrow}
\newcommand{\narrow}{\uparrow}
\newcommand{\lt}{<}
\newcommand{\gt}{>}
\newcommand{\Int}{\intop\limits}
\newcommand{\ures}{\varnothing}
\newcommand{\ekv}{\iff}
\newcommand{\ekviv}{\ekv}
\renewcommand{\epsilon}{\varepsilon}
\newcommand{\eps}{\varepsilon}
%
% ------------  NEW PART DEFS -----------------
%
\newcounter{Szaml}


\theoremstyle{magyar}
\theoremheaderfont{\itshape\bfseries}
\newtheorem{de}{definíció}[section]
\newtheorem{te}{tétel}[section]
\newtheorem{bi}{bizonyítás}[section]
\newtheorem{ko}{következmény}[section]
\newtheorem{me}{megjegyzés}[section]
\newtheorem{al}{állítás}[section]


\newenvironment{korlista}{\begin{enumerate}[\quad1$^\circ$]}{\end{enumerate}}

\newenvironment{biz}{\begin{trivlist}\item\relax\mbox{\textbf{Bizonyítás.\enskip}}\ignorespaces}{\ob\end{trivlist}}
\newenvironment{Biz}{\begin{trivlist}\item\relax\mbox{\textbf{Bizonyítás.\enskip}}\ignorespaces\begin{korlista}}{\ob\end{korlista}\end{trivlist}}
\newenvironment{kov}{\begin{trivlist}\item\relax\mbox{\textbf{Következmény.\enskip}}\ignorespaces}{\end{trivlist}}
\newenvironment{megj}{\begin{trivlist}\item\relax\mbox{\textbf{Megjegyzés.\enskip}}\ignorespaces}{\end{trivlist}}
\newenvironment{Megj}{\begin{megj}\begin{korlista}}{\end{korlista}\end{megj}}
\newenvironment{pl}{\begin{trivlist}\item\relax\mbox{\textbf{Példa.\enskip}}\ignorespaces}{\end{trivlist}}
\newenvironment{Pl}{\begin{pl}\begin{korlista}}{\end{korlista}\end{pl}}
\DeclareMathOperator{\D}{D}
\newenvironment{bizlist}{\setcounter{Szaml}{1}
    \begin{list}{\alph{Szaml})\hfill}
    {\usecounter{Szaml}\setlength{\itemsep}{0pt}
    \setlength{\itemindent}{-\labelsep}
    \setlength{\listparindent}{0pt}}}{\end{list}}




%
% - - - -- - - S E T T I N G S ----------------
%
%\setlength{\parindent}{0pt}
%\setlength{\parskip}{\baselineskip}
\addtolength{\voffset}{-1cm}
\addtolength{\textheight}{2cm}
%\addtolength{\marginparwidth}{-1cm}
\addtolength{\hoffset}{-1cm}
\addtolength{\textwidth}{2cm}
\setlength{\headheight}{23pt}
%
\pagestyle{fancy}

  \renewcommand{\sectionmark}[1]{\markboth{\Roman{section}. tétel\\#1}{}}

\newcommand{\mktoc}{
  \pagenumbering{roman}
  \setcounter{page}{1}
  \lhead{\textbf{\thepage}}
  \cfoot{}
  \tableofcontents
  \newpage
  \lhead{\textbf{\thepage}}%/\pageref{LastPage}}
  \pagenumbering{arabic}
  \setcounter{page}{1}
}


\usepackage{booktabs}


\renewcommand{\sectionmark}[1]{\markboth{\Roman{section}. #1}{}}

\newcommand{\listazjbetu}{
  \renewcommand{\theenumi}{\alph{enumi}}
  \renewcommand{\labelenumi}{(\theenumi)}
}
\newcommand{\listazjromai}{
  \renewcommand{\theenumi}{\alph{enumi}}
  \renewcommand{\labelenumi}{(\theenumi)}
}
\newcommand{\listabetu}{
  \renewcommand{\theenumi}{\alph{enumi}}
  \renewcommand{\labelenumi}{\theenumi}
}
\newcommand{\listaszamkor}{
  \renewcommand{\theenumi}{\alph{enumi}}
  \renewcommand{\labelenumi}{\theenumi$^\circ$}
}
\newenvironment{enumzjromai}{\listazjromai\begin{enumerate}}{\end{enumerate}}
\newenvironment{enumzjbetu}{\listazjbetu\begin{enumerate}}{\end{enumerate}}

\newenvironment{enumzjr}{\begin{enumzjromai}}{\end{enumzjromai}}
\newenvironment{enumzjb}{\begin{enumzjbetu}}{\end{enumzjbetu}}


\DeclareRobustCommand{\tmspace}[3]{%
  \ifmmode\mskip#1#2\else\kern#1#3\fi\relax}
\providecommand*{\negmedspace}{\tmspace-\medmuskip{.2222em}}

\title{Visszázórendszer felhasználói dokumentáció)}
\author{Tóth László Attila (panther@elte.hu)}
\date{}
\begin{document}
  \maketitle
  \mktoc

  \begin{abstract}
    A visszázórendszer egy linuxra írt program, amely néhány fájlt küld át a hálózaton, broadcast UDP csomagokkal. A
    kliens ezt egy fájba menti, mely akár lehet egy speciális eszközfájl is, ezáltal egy merevlemez vagy partíció
    közvetlenül visszatölthető. Ugyanakkor nem feltétlenül az egészet fájlt, mivel mind a szerver, mind a kliens
    ellenőrző összegeket számol, csak az eltérő részek utaznak a hálózaton. A 4GiB-nál nagyobb fájlokat is kezeli a
    rendszer.
    
    Akkor igazán hatékony, ha sok kliens fut egyszerre, így egyszerre több számítógépen lehet a visszatöltést elvégezni.
  \end{abstract}
  
  \section{A program fordítása}
  A program gcc 4.0.3-as fordítóval tesztelt, régebbivel is le kell fordulnia (pl. 3.4.x). A forrás az \texttt{src}
  könyvtárban található, oda belépve egy \texttt{make} parancsot kell beírni a parancssorba. A program többszálú, ezért
  a \texttt{pthread} függvényeket használja.

  \section{A program indítása}
  A visszázórendszer két részből áll, a szerverből, ahol a visszázandó fájlok (image-ek) találhatók, valamint a
  kliensből, ami a visszatöltést végzi. Az adat broadcast UDP-n utazik a hálózaton, ezért az a célszerű, ha egyszerre
  több (sok) kliens fut, és ugyanazt töltik. A csomagok 1024 bájtnyi adatot és a fejlécet tartalmazzák.

  A lefordított bináris fájlok a \texttt{bin} könyvtárban vannak. Mindegyik egy-egy konfigurációs állományt és esetleg
  további kapcsolókat vár.

  \subsection{A szerver}
  A szerver indítása a \texttt{yars-server} paranccsal történik, az alapértelmezett beállítási fájl a
  \texttt{/etc/vissza/server.conf}. Az ettől eltérő állományt a \texttt{-f FILE} opcióval lehet megadni.
  
  \subsubsection{A konfigurációs állomány}
  A következő beállítások adhatók meg szervernek, mindegyik szükséges. A példákban az egyenlőségjel nem kötelező, elég
  egy vagy több szóköz a név-érték párok között.
  
  \texttt{broadcast = hálózat}, például \texttt{broadcast 10.0.1.255}. Ez a sor mondja meg a broadcast hálózati
  címet. Javasolt egy belső hálózati cím használata (pl. a 10.0.0.0/8-as tartomány egy része).


  \texttt{port 8766} Ezen a porton fogadja a szerver a kéréseket, hogy melyik fájl melyik részét kell elküldeni a
  hálózaton. TCP-t használ. Ez az alapértelmezett érték.
  
  \texttt{info-port 8767} Ezen a TCP porton lehet lekérdezni valamelyik image adatait.
  
  \texttt{broadcast-port = 8769} A szerver erre a portra küldi a broadcast UDP datagramokat. A kliens ezen a porton
  hallgatózik. 

  \texttt{names = LIST} Itt a \texttt{LIST} egy szóközökkel elválasztott lista, minden szó egy-egy visszatölthető fájlt
  jelöl. Amely nevek nem érvényesek, mert pl. nincsen hozzájuk tartozó fájl, azok nem kerülnek be a szerver listájába,
  vagyis a kliens nem próbálja meg letölteni.
  
  \texttt{IMAGE-NAME = FILE} Az előző listában szereplő nevek (\texttt{IMAGE-NAME}) egyike. A  korábbi foglalt neveket
  leszámítva bármilyen szóközt nem tartalmazó név érvényes. A \texttt{FILE} a szerver számítógépen adja meg az elérési
  utat. A sor összmérete 3 KiB lehet.

  \subsubsection{Megjegyzések}
  A szerver a \texttt{broadcast-port} beállítást használja fel arra, hogy honnan küldje a broadcast csomagokat. A
  megadott kapuszám felettit használja, azza a példában a 8770-eset.
  
  \subsection{A kliens}
  A kliens indítása a \texttt{yars-client} paranccsal történik, az alapértelmezett beállítási fájl a
  \texttt{/etc/vissza/client.conf}. Az ettől eltérő állományt a \texttt{-f FILE} opcióval lehet megadni.
  
  \subsubsection{A konfigurációs állomány}
  A következő beállítások adhatók meg kliensnek, mindegyik szükséges. A példákban az egyenlőségjel nem kötelező, elég
  egy vagy több szóköz a név-érték párok között.
  
  \texttt{server-ip = 10.0.1.100} Ez a szerver IP címe.
  
  \texttt{server-port = 8767} A szerver TCP \texttt{info-port}jával azonos beállítás, ahonnan az image-ek adatait lehet
  lekérdezni.
  
  \texttt{server-req-port 8766} A szerver ezen a porton fogadja a kéréseket, hogy az image melyik részét kell elküldeni.

  
  \texttt{broadcast-port = 8769} A szerver erre a portra küldi el a csomagokat.

  \subsubsection{Parancssori opciók}
  A kliens indításához a legtöbb opció elengedhetetlen, ezeket jelzi is, ha nincsenek megadva.
  
  \texttt{-f, -{}-config-file=FILE} A beállításokat tartalmazó fájl, az alapértelmezett\\ a \texttt{/etc/vissza/client.conf}
  
  \texttt{-t, -{}-targetfile=FILE} Az a fájlnév, amibe menti a letöltött adatokat.

  \texttt{-m, -{}-image=IMAGE-NAME} A visszatöltendő image neve.

  \texttt{-h, -{}-help} Az opciók rövid leírása (angolul).
  
  \texttt{-d}  Debug mód.

  \subsubsection{Megjegyzések}
  A kliens egy-egy pontot ír ki a képernyőre minden egyes megérkezett csomag esetén. Ha elveszik egy-egy csomag, majd
  későbbi indexű következik, akkor \texttt{|2|} formában jelzi, mennyi veszett el. Itt most 2 darab.
  
  A kliens csak annyi részt akar letölteni a szerverről, amennyi nincs meg neki. Ez 1 MiB-os részeket jelöl, kivéve az
  utolsót, mert az lehet kisebb is. Ekkora részekről készít ellenőrző összeget (MD5), akárcsak a szerver.
\end{document}

% Local Variables:
% fill-column: 120
% TeX-master: t
% End:
